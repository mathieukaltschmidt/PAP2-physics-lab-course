%kompiliert mit XeLaTeX

\documentclass[DIV=13]{scrartcl}
%Laden des Mathe-Pakets
\usepackage{amsmath}
%Spracheinstellung & Microtypografie
\usepackage{fontspec} 
\usepackage{polyglossia}
\setmainlanguage{german}
\usepackage{microtype}
\setmainfont{Linux Libertine O}
%Erstellen der Kopfzeile
\usepackage[headsepline]{scrpage2}
\pagestyle{scrheadings}
\setkomafont{pageheadfoot}{\normalfont}
\ihead{Versuch 234: Lichtquellen}
\ohead{Mathieu Kaltschmidt}

\usepackage{listings}
\usepackage{xcolor}
\usepackage{float}
\usepackage{graphicx}
\usepackage{siunitx}

\definecolor{mygreen}{rgb}{0,0.6,0}
\definecolor{mygray}{rgb}{0.5,0.5,0.5}

\lstset{backgroundcolor=\color{gray!8},basicstyle=\footnotesize,breakatwhitespace=false,breaklines=true,captionpos=b,commentstyle=\color{blue!60!purple!60},stringstyle=\color{red!80},  frame=single,keywordstyle=\color{mygreen},language={Python},numbers=left,numbersep=5pt,numberstyle=\tiny\color{mygray},showspaces=false,showstringspaces=false,showtabs=false,stepnumber=2	}
\begin{document}
\section*{Auswertung der Messergebnissse}
\subsection*{Teil 1: Auswertung des Sonnenspektrums}
\lstinputlisting[firstline=10,lastline=18]{V234Lichtquellen.py.html}
\lstinputlisting[firstline=23,lastline=30]{V234Lichtquellen.py.html}
\subsubsection*{Vergleich der aufgenommenen Spektren}
\lstinputlisting[firstline=37,lastline=50]{V234Lichtquellen.py.html}
\subsubsection*{Berechnung der Absorption}
\lstinputlisting[firstline=57,lastline=69]{V234Lichtquellen.py.html}
\subsubsection*{Analyse der Fraunhoferlinien (am Bsp. des Himmelslichts,weil keine direkte Sonnenmessung möglich war)}
\lstinputlisting[firstline=76,lastline=86]{V234Lichtquellen.py.html}
\lstinputlisting[firstline=91,lastline=113]{V234Lichtquellen.py.html}
\lstinputlisting[firstline=118,lastline=138]{V234Lichtquellen.py.html}
\section*{Teil 2: Auswertung des Natriumspektrums}
\lstinputlisting[firstline=145,lastline=156]{V234Lichtquellen.py.html}
\lstinputlisting[firstline=161,lastline=183]{V234Lichtquellen.py.html}
\lstinputlisting[firstline=188,lastline=193]{V234Lichtquellen.py.html}
\lstinputlisting[firstline=198,lastline=218]{V234Lichtquellen.py.html}
\lstinputlisting[firstline=223,lastline=228]{V234Lichtquellen.py.html}
\lstinputlisting[firstline=233,lastline=245]{V234Lichtquellen.py.html}
\lstinputlisting[firstline=250,lastline=255]{V234Lichtquellen.py.html}
\subsection*{Teil 3: Zuordnung der Linien zu den Serien}
\subsubsection*{Erste Nebenserie $ md \rightarrow 3p$ }
\lstinputlisting[firstline=264,lastline=277]{V234Lichtquellen.py.html}
\subsubsection*{Zweite Nebenserie $ ms \rightarrow 3p$ }
\lstinputlisting[firstline=284,lastline=299]{V234Lichtquellen.py.html}
\subsubsection*{Hauptserie $ mp \rightarrow 3s$ }
\lstinputlisting[firstline=306,lastline=317]{V234Lichtquellen.py.html}
\subsection*{Teil 4: Bestimmung der Serienenergien und der l-abhängigen Korrekturfaktoren}
\lstinputlisting[firstline=324,lastline=327]{V234Lichtquellen.py.html}
\lstinputlisting[firstline=332,lastline=357]{V234Lichtquellen.py.html}
\lstinputlisting[firstline=362,lastline=374]{V234Lichtquellen.py.html}
\lstinputlisting[firstline=379,lastline=382]{V234Lichtquellen.py.html}
\lstinputlisting[firstline=387,lastline=413]{V234Lichtquellen.py.html}
\lstinputlisting[firstline=418,lastline=432]{V234Lichtquellen.py.html}

\end{document}